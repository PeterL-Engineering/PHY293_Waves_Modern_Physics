\documentclass{article} % For LaTex2e
\usepackage{iclr2022_conference,times}
% Optional math commands from https://github.com/goodfeli/dlbook_notation.
\input{math_commands.tex}

%######## PHY293: Uncomment your submission name
\newcommand{\phyname}{ - Lab Report}
%\newcommand{\phyname}{Prograss Report}
%\newcommand{\phyname}{Final Report}

%######## PHY293: Put your Group Number here
%\newcommand{\gpnumber}{40}

\usepackage{hyperref}
\usepackage{xcolor}
\usepackage[normalem]{ulem}
\usepackage{url}
\usepackage{graphicx}
\usepackage{placeins}
\usepackage{float}
\usepackage{tikz}
\usepackage{multicol}
\usepackage{makecell}
\usepackage[table]{xcolor}

%######## PHY293: Put your project Title here
\title{The Output Resistance of A Power Supply \\
An Introductory Uncertainty Analysis Exercise}

%######## PHY293: Put your names, student IDs and Emails here
\author{Karys Littlejohns\\
Student\# \\
karys.littlejohns@mail.utoronto.ca
\And
Peter Leong \\
Student\# 1010892955 \\
peter.leong@mail.utoronto.ca \\
\AND
}

% The \author macro works with any number of authors. There are two commands
% used to separate the names and addresses of multiple authors: \And and \AND.
%
% Using \And between authors leaves it to \LaTex{} to determine where to break
% the lines. Using \AND forces a linebreak at that point. So, if \LaTex{}
% puts 3 of 4 authors names on the first line, and the last on the second
% line, try using \AND instead of \And before the third author name.

\newcommand{\fix}{\marginpar{FIx}}
\newcommand{\new}{\marginpar{NEW}}

\iclrfinalcopy 
%######## PHY293: Document starts here
\begin{document}

\maketitle

\vspace{-6ex}

\begin{abstract}

----Total Pages: \pageref{last_page}
\end{abstract}

\vspace{2ex}

\section{Introduction}

Sources of electricity produce a voltage potential across their terminals called an electromotive force (\textit{emf}) or an open circuit voltage, $V_{\infty}$.
In practice, when a closed circuit is made, a current \textit{I} will be drawn and the voltage at the terminals, \textit{V} called \textit{terminal voltage} will fall below $V_{\infty}$.
For most cases, many powers sources will exhibit a linear variation of R for small current values, with nonlinear behaviour at higher currents.
The linear part of the curve can be described by:
\[
V = V_{\infty} - RI
\]
where R is the \textit{output resistance of the powers source}.
In this linear regime, according to Thevenin's theorem, the power source is completely represented by the equivalent circuit shown below.

\begin{figure}[htbp]            % h=here, t=top, b=bottom, p=page float
  \centering
  \includegraphics[width=0.65\linewidth]{Figs/power_source.png}
  \caption{Equivalent circuit of an electric power source.}
  \label{fig:power_source}
\end{figure}

The output resistance (\textit{R}) can be determined by attaching different external resistances of the load ($R_{l}$) to the power source, and measuring the current and voltage with a multimeter.
\ref{fig:two_circuits} shows two possible ways of doing this. Both would be equivalent \textbf{if} the multimeter were ideal.
However, in this exercuse we will measure with real, not ideal, multimeters.

\begin{figure}[htbp]            % h=here, t=top, b=bottom, p=page float
  \centering
  \includegraphics[width=0.65\linewidth]{Figs/two_circuits.png}
  \caption{Possible circuits for determining the output resistance of a power source.}
  \label{fig:two_circuits}
\end{figure}

\section{Pre-Lab Exercises}

Without connecting circuits and making measurements, we can expect the the voltmeter and ammeter readings to differ between the two options due to current and voltage leakage.
More specifically, in option 1, we can expect the current to differ as the voltmeter allows some current to pass through. The amount of leakage will depend on the load resistance.
Similarly, in option 2, we can expect that the voltage will differ as the ammeter will draw some voltage. Its leakage will too depend on the load resistance.

To calculate the internal resistances of the voltmeter and the ammeter we can use basic current and voltage division.
For the voltmeter, we can calculate the current that we expect to pass through the ammeter and solve for the voltmeter resistance.

\[
I_{ammeter} = I \frac{R_{voltmeter}}{R_{load} + R_{voltmeter}}
\]

\[
I(R_{load} + R_{voltmeter}) = R_{voltmeter}
\]

\[
R_{voltmeter} = \frac{I R_{load}}{1 - I}
\]

We can calculate the internal resistance of the ammeter similarly by using voltage division for the second circuit:

\[
I_{ammeter} = \frac{V}{R_{ammeter} + R_{load}}
\]

\[
I(R_{ammeter} + R_{load}) = V
\]

\[
R_{ammeter} = \frac{V}{I_{ammeter}} - R_{load}
\]

\section{The Experiment}

We began by measuring the resistance values of the provided resistors. 
For the subsequent circuit experiments, we selected the two highest and two lowest resistance values to represent the load resistance conditions for the voltmeter and ammeter configurations, respectively. 
The measured values with their associated uncertainties are presented below.

\label{table_resistor_values}
\begin{table}[htbp]
\centering
\caption{Measured Resistor Values and Uncertainties}
\begin{tabular}{|c|c|c|c|c|c|c|}
\hline
 & $R_{l1}$ & $R_{l2}$ & $R_{l3}$ & $R_{l4}$ & $R_{l5}$ & $R_{l6}$ \\
\hline
Value ($\Omega$) & 100.32 & 219.91 & 461.3 & 2.6960 k & 26.814 k & 101.57 k \\
\hline
Uncertainty ($\Omega$) & $\pm$0.25 & $\pm$0.49 & $\pm$1.4 & $\pm$5.9 & $\pm$59 & $\pm$250 \\
\hline
\end{tabular}
\end{table}

INCLUDE DIAGRAM/SKETCH OF CIRCUIT OPTION 1

\begin{table}[htbp]
\centering
\caption{Circuit 1 Readings \& Results}
\begin{tabular}{|c|c|c|c|c|c|c|c|}
\hline
\makecell{Resistance \\ $R_{li}$ ($\Omega$)} & \makecell{Uncertainty \\ $\Delta R_{li}$ ($\Omega$)} & \makecell{Voltage \\ $V$ (V)} & \makecell{Uncertainty \\ $\Delta V$ (V)} & \makecell{Current \\ $I$ (mA)} & \makecell{Uncertainty \\ $\Delta I$ (mA)} & \makecell{Ammeter Resistance \\ $R_A$ ($\Omega$)} & \makecell{Uncertainty \\ $\Delta R_A$ ($\Omega$)} \\
\hline
100.32 & $\pm$0.25 & 6.501 & $\pm$0.005 & 63.67 & $\pm$0.18 & & \\
\hline
219.91 & $\pm$0.49 & 6.501 & $\pm$0.005 & 29.315 & $\pm$0.064 & & \\
\hline
26.814 k & $\pm$0.059 k & 6.501 & $\pm$0.005 & 0.241 & $\pm$0.051 & & \\
\hline
101.57 k & $\pm$0.25 k & 6.501 & $\pm$0.005 & 0.063 & $\pm$0.005 & & \\
\hline
\cellcolor{gray!50} & \cellcolor{gray!50} & \cellcolor{gray!50} & \cellcolor{gray!50} & \cellcolor{gray!50} & Average: &  &  \\
\hline
\end{tabular}
\end{table}

INCLUDE DIAGRAM/SKETCH OF CIRCUIT OPTION 2

\begin{table}[htbp]
\centering
\caption{Circuit 2 Readings \& Results}
\begin{tabular}{|c|c|c|c|c|c|c|c|}
\hline
\makecell{Resistance \\ $R_{li}$ ($\Omega$)} & \makecell{Uncertainty \\ $\Delta R_{li}$ ($\Omega$)} & \makecell{Voltage \\ $V$ (V)} & \makecell{Uncertainty \\ $\Delta V$ (V)} & \makecell{Current \\ $I$ (mA)} & \makecell{Uncertainty \\ $\Delta I$ (mA)} & \makecell{Voltmeter Resistance \\ $R_V$ ($\Omega$)} & \makecell{Uncertainty \\ $\Delta R_V$ ($\Omega$)} \\
\hline
100.32 & $\pm$0.25 & 6.386 & $\pm$0.005 & 63.60 & $\pm$0.18 & & \\
\hline
219.91 & $\pm$0.49 & 6.448 & $\pm$0.005 & 29.322 & $\pm$0.064 & & \\
\hline
26.814 k & $\pm$0.059 k & 6.501 & $\pm$0.005 & 0.243 & $\pm$0.051 & & \\
\hline
101.57 k & $\pm$0.25 k & 6.501 & $\pm$0.005 & 0.065 & $\pm$0.005 & & \\
\hline
\cellcolor{gray!50} & \cellcolor{gray!50} & \cellcolor{gray!50} & \cellcolor{gray!50} & \cellcolor{gray!50} & Average: &  &  \\
\hline
\end{tabular}
\end{table}


\section{Discussion}

\section{Conclusion}

\label{last_page}

\newpage
%\bibliographystyle{iclr2022_conference}
%\bibliography{PHY293_Resistance_Exercise}

\end{document}