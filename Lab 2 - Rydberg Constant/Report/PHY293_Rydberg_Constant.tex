\documentclass{article} % For LaTex2e
\usepackage{iclr2022_conference,times}
% Optional math commands from https://github.com/goodfeli/dlbook_notation.
\input{math_commands.tex}

%######## PHY293: Uncomment your submission name
\newcommand{\phyname}{ - Lab Report}
%\newcommand{\phyname}{Progress Report}
%\newcommand{\phyname}{Final Report}

%######## PHY293: Put your Group Number here
%\newcommand{\gpnumber}{40}

\usepackage{hyperref}
\usepackage{xcolor}
\usepackage[normalem]{ulem}
\usepackage{url}
\usepackage{graphicx}
\usepackage{placeins}
\usepackage{float}
\usepackage{tikz}
\usepackage{multicol}

%######## PHY293: Put your project Title here
\title{Real-Time Neural Signal Filtering via \\
Hodgkin-Huxley Simulation Models}

%######## PHY293: Put your names, student IDs and Emails here
\author{Karys Littlejohns\\
Student\# \\
karys.littlejohns@mail.utoronto.ca
\And
Peter Leong \\
Student\# 1010892955 \\
peter.leong@mail.utoronto.ca \\
\AND
}

% The \author macro works with any number of authors. There are two commands
% used to separate the names and addresses of multiple authors: \And and \AND.
%
% Using \And between authors leaves it to \LaTex{} to determine where to break
% the lines. Using \AND forces a linebreak at that point. So, if \LaTex{}
% puts 3 of 4 authors names on the first line, and the last on the second
% line, try using \AND instead of \And before the third author name.

\newcommand{\fix}{\marginpar{FIx}}
\newcommand{\new}{\marginpar{NEW}}

\iclrfinalcopy 
%######## PHY293: Document starts here
\begin{document}

\maketitle

\vspace{-6ex}

\begin{abstract}

----Total Pages: \pageref{last_page}
\end{abstract}

\vspace{2ex}

\begin{multicols}{2}

\section{Introduction}

\section{Bohr's Hydrogen Atom Theory}

Bohr's theory of the hydrogen an atom was an early model to include quantum concepts, namely: quantized energy levels.
More importantly, his model enabled him to derive a numerical formula for the Rydberg constant based on the three key postulates.
Firslty, within this model, electrons revolve in specific orbits around the nucleus as defined by their energy level.
Secondly, the angular momentum $L$ of an electron is also quantized and is given by the formula:

\[
L = M_{e}vr_{n} = n \frac{h}{2pi}
\]

where $m_{e}$ is the electron's mass, $r_{n}$ is the radius of the $n$th orbit, and $h$ is Planck's constant.
Lastly, electrons move between quantized energy levels whereby they produce a signle photon of light.
Using these postulates, Bohr derived the formula for the Rydberg constant using the $n$th level energy of an electron:

\[
E_n = -\frac{2\pi^2 m_e e^4}{(4\pi\epsilon_0)^2 h^2} \frac{1}{n^2} = -\frac{hc R_H}{n^2}
\]

The energy emitted by a photon during transition from $n_i$ to $n_f$ is then:

\[
E_{\text{photon}} = hc R_H \left( \frac{1}{n_f^2} - \frac{1}{n_i^2} \right) 
\]

We can thus derive the famous Rydberg formula given below:

\[
\frac{1}{\lambda} = R_H \left( \frac{1}{n_f^2} - \frac{1}{n_i^2} \right)
\]

Although Bohr made great strides in understanding the atomic model, there are several key limitations in his theory's application.
The greatest of which is that Bohr's model fails for anything other than a single-electron atom.
Further, his theory cannot explain the intensity of emission spectra nor why some spectral lines are split into two.

\section{Equipment \& Procedure}

To observe these phenomena, key equipment including a spectrometer and a sodium lamp were used.
Additionally, the Hartmann Relation method was used to tune and calibrate the experiment apparatus.
The subsequent outline the equipments' function, calibration and experimental procedures in further detail.

\subsection{Station Apparatus \& Instruments}

The Rydberg experiment station's primary apparatus is a triprism spectrometer capable of resolving to 0.1 nm.


\subsection{Calibration with the Hartmann Relation Method}

The main objective of calibrating the spectrometer is determinign the relationship between the wavelength $\lambda$ of a spectra line and the scale reading $y$ on the instrument.
This enables the conversion of an unknown line scale reading to a numerical wavelength.
More specifically, the Hartmann relation method was used to determine this relationship via equation \textbf{REFERENCE EQUATION}
\[
y = \frac{m}{\lambda - \lambda_0} + b
\]

where $\lambda_0$, $m$, and $b$ are constants.
To determine the values for $m$ and $b$, $y$ was plot against $(\lambda - \lambda_0)^-1$.
Based on the given Helium and hydrogen strong lines, corresponding values for $y$ were obtained.
\textbf{INCLUDE APPENDIX ON UNCERTAINTY CALCULATION}

\subsection{Identification of the Unknown Gas}

The calibration gas tube was replaced with the unknown gas.
For observable spectral lines in the unknown gas spectrum, values for $y$ were obtained and are summarized in \textbf{REFERENCE FUTURE TABLE IN RESULTS SECTION}
The spectra were then compared to the given emission spectra of various gases as shown in \textbf{REFERENCE FIGURE OF GAS SPECTRUM}

\subsection{The Rydberg Constant}

Using the wavelengths of the spectral lines of hydrogen, we can calculate the Rydberg constant $R_H$ by appling \textbf{EQ 3? RYDBERG CONSTANT EQUATION}
We can then easily convert our determined value of $R_H$ into $R_{EH}$ in electronvolts.
\textbf{INCLUDE APPENDIX ON UNCERTAINTY CALCULATION}

\subsection{Calculating the Separation of Spectral Lines in the Yellow Doublet of Sodium}

First the power supply for the discharge tubes was turned off and the sodium lamp was turned on.
The separation between the two yellow lines was measured once the colour of the discharge became yellow \textbf{EXPRESS IN NM, CM-1, AND EV}

The purpose of calculating this separation is to adjust the optical devices to the best resolution.
\textbf{CALCULATE THE WAVELENGTHS OF THE TWO YELLOW LINES IN THE SPECTRUM OF SODIUM VAPOR}.
\textbf{CONCLUDE ON THE ACCURACY OF MEASUREMENTS WITH THE SPECTROMETER IN THIS EXPERIMENT AND ACCURACY OF CALIBRATION PERFORMED IN PART 1}

\section{Results \& Discussion}

\subsection{Vernier Scale Readings}

\subsection{Rydberg Constant Results}

\subsection{Unknown Gas Identification}

\subsection{Spectrometer Resolution via Sodium Spectrum}

\section{Conclusion}

\label{last_page}

\newpage
\bibliographystyle{iclr2022_conference}
\bibliography{PHY293_Proposal_Ref}

\end{multicols}
\end{document}