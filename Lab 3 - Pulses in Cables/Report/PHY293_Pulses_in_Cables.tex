\documentclass{article} % For LaTex2e
\usepackage{iclr2022_conference,times}
% Optional math commands from https://github.com/goodfeli/dlbook_notation.
\input{math_commands.tex}

%######## PHY293: Uncomment your submission name
\newcommand{\phyname}{ - PRA0104: 24 September 2025 (12PM - 3PM)}
%\newcommand{\phyname}{Progress Report}
%\newcommand{\phyname}{Final Report}

%######## PHY293: Put your Group Number here
%\newcommand{\gpnumber}{40}

\usepackage{hyperref}
\usepackage{xcolor}
\usepackage[normalem]{ulem}
\usepackage{url}
\usepackage{graphicx}
\usepackage{placeins}
\usepackage{float}
\usepackage{tikz}
\usepackage{multicol}

%######## PHY293: Put your project Title here
\title{Real-Time Neural Signal Filtering via \\
Hodgkin-Huxley Simulation Models}

%######## PHY293: Put your names, student IDs and Emails here
\author{Karys Littlejohns\\
Student\# \\
karys.littlejohns@mail.utoronto.ca
\And
Peter Leong \\
Student\# 1010892955 \\
peter.leong@mail.utoronto.ca \\
\AND
}

% The \author macro works with any number of authors. There are two commands
% used to separate the names and addresses of multiple authors: \And and \AND.
%
% Using \And between authors leaves it to \LaTex{} to determine where to break
% the lines. Using \AND forces a linebreak at that point. So, if \LaTex{}
% puts 3 of 4 authors names on the first line, and the last on the second
% line, try using \AND instead of \And before the third author name.

\newcommand{\fix}{\marginpar{FIx}}
\newcommand{\new}{\marginpar{NEW}}

\iclrfinalcopy 
%######## PHY293: Document starts here
\begin{document}

\maketitle

\vspace{-6ex}

\begin{abstract}
This project addresses the challenge of automated colourization for 256$\times$256 grayscale images using a dataset of 12,600 image pairs, balanced across human subjects, 
animals, and natural scenery. We frame colourization as a supervised learning problem in the CIELAB colour space, where a model predicts chrominance channels ($a^*$, $b^*$) 
from the luminance channel ($L^*$). A shallow convolutional neural network (CNN) provides the baseline performance, while our primary solution employs a deeper convolutional 
encoder-decoder architecture. This design captures high-level semantic features and spatial context, addressing limitations of shallow networks in perceptual realism.
%######## PHY293: Do not change the next line. This shows your Main body page count.
----Total Pages: \pageref{last_page}
\end{abstract}

\vspace{2ex}

\section{Introduction}

Standard AC and DC circuit theory makes the assumption that the speed of information propagation in a circuit is ideal and is infinite.
Thus, if a change occurs anywhere along the circuit, then the rest of the circuit will respond instantaneously to that change.
In reality, pulses are limited by the physical properties of the cable insulator, and do not travel nearly as fast as the speed of light.

\section{Background Information}



\subsection{The Ideal Transmission Line}

Transmission lines can be abstracted as discrete repeated networks of inducts (L) and capcitors (C) that can store and transmit electric and magnetic energy.
The most foundational block of an ideal transmission line is an LC element of lenght $dx$ and zero electric resistance \textbf{REFERENCE IMAGE OF TRANSMISSION LINE}

\textbf{INSERT IMAGE OF TRANSMISSION LINE CIRCUIT HERE}

We can represent these as waves by first analysing the rates of change for both current I and voltage V take into account the self inductance of the element ($L_0dx$) and its capacitance ($C_0dx$)

\[
\frac{\partial V}{\partial x} = -L_0\frac{\partial I}{\partial t}, \quad \frac{\partial I}{\partial x} = -C_0\frac{\partial V}{\partial t}
\]

From these rates of change, we can obtain the wave equations for voltage and current, and the velocity of wave propagation:

\[
\frac{\partial^2 V}{\partial x^2} = L_0C_0\frac{\partial^2 V}{\partial t^2}, \quad \frac{\partial^2 I}{\partial x^2} = L_0C_0\frac{\partial^2 I}{\partial t^2}
\]
\[
\nu^2 = \frac{}{L_0 C_0}
\]

We can integrate the voltage drop along an infinitesimal unit of length with a solution of:
\[
V_+ = \nu L_0I_+
\]
where the subscript + means the positive direction of wave propagation.
The ration then of voltage to current is called the characteristic impedance of the line $Z_0$:
\[
\frac{V_+}{I_+} = Z_0 = \nu L_0 = \sqrt{\frac{L_0}{C_0}}
\]

\subsection{Coaxial Cables}

Consider two conductor cables, separated by a dielectric material \textbf{REFERENCE FIGURE OF CROSS SECTIONAL CABLE} with a continuous sitribution of LC elements. 
Inductance per unit length can be written as:
\[
L_0 = \frac{\mu}{2\pi}\ln\frac{R_2}{R_1}
\] 

\textbf{INSERT COAXIAL CABLE CROSS SECTION HERE}

where $R_1$ and $R_2$ are the radii of the inner and outer conductors, respectively; $\mu$ is the magnetic permeability of the dielctric.
Capcitance per unit length is given by:
\[
C_0 = \frac{2\pi\epsilon}{ln\frac{R_2}{R_1}}
\]
where $\epsilon$ is the permittivity of the dielectric.
Taking the product of equations 10 and 11 it can be verified that
\[
\frac{1}{L_0C_0} = \frac{1}{\epsilon\mu} = \nu^2
\]

Knowing the velocity of signal travel in a certain coaxial cable, we can accurately calculate the travel time of an electrical signal along a given length.
A useful visual analogy of a pulse traveling in the cable is that of a wave moving along a rope. 
We shall see that the action of the pulse at the end of the cable has a strong analogy with what happens at the fixed end of an oscillating rope, suggesting that the math might be similar.
In the laboratory we shall see these effects by using pulses of duration ~ 10-8 seconds = 10ns, since this corresponds to a “length” of ~3m.
“Pulse length” is the real physical length of the pulse in the cable, so we can use the “length” of the pulse in either time units or length units.

\subsection{The Load Effect}

To determine the effect of a load with impedance $Z_L$ at the end of the cable we can solve the boundary conditions that must be satisfied by a wave travilling in the positive + and negative + at the boundary.
Inside the cable the impedance remains the same
\[
\frac{V_+}{I_+} = Z_0 = - \frac{V_-}{I_-}
\]
at the end of the cable the load $Z_L$ sees teh combined voltage $V = V_+ + V_-$ and current $I = I_+ + I_-$ to give:
\[
Z_L = \frac{V}{I}
\]
These equations can be used to define a reflection coefficient $r = \frac{V_-}{V_+}$ and a transmission coefficient $t = \frac{V}{V_+}$

\begin{align*}
    r = 
\end{align*}

\section{Methods \& Procedures}

\subsection{Exercise 1}

\subsection{Exercise 2}

\section{Conclusion}

\label{last_page}

\newpage
% \bibliographystyle{iclr2022_conference}
% \bibliography{PHY293_Proposal_Ref}

\end{document}