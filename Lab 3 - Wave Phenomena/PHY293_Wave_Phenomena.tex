\documentclass{article} % For LaTex2e
\usepackage{iclr2022_conference,times}
% Optional math commands from https://github.com/goodfeli/dlbook_notation.
\input{math_commands.tex}

%######## PHY293: Uncomment your submission name
%\newcommand{\chename}{ - Project Proposal}
%\newcommand{\chename}{Progress Report}
%\newcommand{\chename}{Final Report}

%######## PHY293: Put your Group Number here
%\newcommand{\gpnumber}{40}

\usepackage{hyperref}
\usepackage{xcolor}
\usepackage[normalem]{ulem}
\usepackage{url}
\usepackage{graphicx}
\usepackage{placeins}
\usepackage{float}
\usepackage{tikz}
\usepackage{multicol}

%######## PHY293: Put your project Title here
\title{Real-Time Neural Signal Filtering via \\
Hodgkin-Huxley Simulation Models}

%######## PHY293: Put your names, student IDs and Emails here
\author{Karys Littlejohns\\
Student\# \\
karys.littlejohns@mail.utoronto.ca
\And
Peter Leong \\
Student\# 1010892955 \\
peter.leong@mail.utoronto.ca \\
\AND
}

% The \author macro works with any number of authors. There are two commands
% used to separate the names and addresses of multiple authors: \And and \AND.
%
% Using \And between authors leaves it to \LaTex{} to determine where to break
% the lines. Using \AND forces a linebreak at that point. So, if \LaTex{}
% puts 3 of 4 authors names on the first line, and the last on the second
% line, try using \AND instead of \And before the third author name.

\newcommand{\fix}{\marginpar{FIx}}
\newcommand{\new}{\marginpar{NEW}}

\iclrfinalcopy 
%######## PHY293: Document starts here
\begin{document}

\maketitle

\vspace{-6ex}

\begin{abstract}
This project addresses the challenge of automated colourization for 256$\times$256 grayscale images using a dataset of 12,600 image pairs, balanced across human subjects, 
animals, and natural scenery. We frame colourization as a supervised learning problem in the CIELAB colour space, where a model predicts chrominance channels ($a^*$, $b^*$) 
from the luminance channel ($L^*$). A shallow convolutional neural network (CNN) provides the baseline performance, while our primary solution employs a deeper convolutional 
encoder-decoder architecture. This design captures high-level semantic features and spatial context, addressing limitations of shallow networks in perceptual realism.
%######## PHY293: Do not change the next line. This shows your Main body page count.
----Total Pages: \pageref{last_page}
\end{abstract}

\vspace{2ex}

\begin{multicols}{2}

\section{Introduction}

\section{Scope \& Feasibility}

Begin by solving Hudgkin-Huxley equations using Euler's and Improved Euler's Method (since these are the methods we have learned thus far from ESC103)
then expand and solve using Runge-Kutta methods once they are covered in PHY293. At a high-level, these solvers will generate synthetic neural data through
the Hodgkin-Huxley equations which are a set of non-linear ordinary differential equations that describe how ion channels create action potential.

Following this we will design and implement a digital filter to isolate the spike band, and create a spike detection algorithm based upon an adaptive threshold
derived from the signal's noise floor.

\section{Technical Background}

\section{Conclusion}

\label{last_page}

\newpage
\bibliographystyle{iclr2022_conference}
\bibliography{PHY293_Proposal_Ref}

\end{multicols}
\end{document}